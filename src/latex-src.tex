\documentclass[a4paper,12pt]{article}
\usepackage[spanish]{babel}
\begin{document}
\title{Título del artículo}
\author{Nombre y Apellido \\
         Técnicas Experimentales~\footnote{Universidad de La Laguna}
       }
\date{\today}
\maketitle
\begin{abstract}
 En \LaTeX{}~\cite{Lam:86} es sencillo escribir expresiones
 matemáticas como $a=\sum_{i=1}^{10} {x_i}^{3}$
 y deben ser escritas entre dos símbolos \$.
 Los super índices se obtienen con el símbolo \^{}, y
 los subíndices con el símbolo \_.
 Por ejemplo: $x^2 \times y^{\alpha + \beta}$.
 También se pueden escribir fórmulas centradas:
 \[h^2=a^2 + b^2 \]
\end{abstract}

\section{Primera sección}
 Si simplemente se desea escribir texto normal en LaTeX,
 sin complicadas f\'ormulas matem\'aticas o efectos especiales
 como cambios de fuente, entonces simplemente tiene que escribir
 en espa\~nol normalmente.\par
 Si desea cambiar de párrafo ha de dejar una línea en blanco o bien
 utilizar el comando.\par
 No es necesario preocuparse de la sangría de los párrafos:
 todos los párrafos se sangrarán automáticamente con la excepción
 del primer párrafo de una sección.

 Se ha de distinguir entre la comilla simple ‘izquierda’
 y la comilla simple ‘derecha’ cuando se escribe en el ordenador.
 En el caso de que se quieran utilizar comillas dobles se han de
 escribir dos caracteres ‘comilla simple’ seguidos, esto es,
 ‘‘comillas dobles’’.

 También se ha de tener cuidado con los guiones: se utiliza un 
 guión para la separación de sílabas, mientras que se utilizan
 tres guiones seguidos para producir un guión de los que se usan
como signo de puntuación --- como en esta oración.
\begin{thebibliography}{00}
 \bibitem{Lam:86}
  Lamport, Leslie.
  TLA in pictures.
  \emph{IEEE Transactions on Software Engineering},
  21(9), 768-775.
  (1995)
\end{thebibliography}
\end{document}