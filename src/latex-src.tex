\documentclass[a4paper,10pt]{letter}
\begin{document}
 Si simplemente se desea escribir texto normal en LaTeX,
 sin complicadas f ́ormulas matem ́aticas o efectos especiales
 como cambios de fuente, entonces simplemente tiene que escribir
 en espa~nol normalmente.
 Si desea cambiar de p ́arrafo ha de dejar una l ́ınea en blanco o bien
 utilizar el comando \par.
 No es necesario preocuparse de la sangr ́ıa de los p ́arrafos:
 todos los p ́arrafos se sangrar ́an autom ́aticamente con la excepci ́on
 del primer p ́arrafo de una secci ́on.
 Se ha de distinguir entre la comilla simple ‘izquierda’
 y la comilla simple ‘derecha’ cuando se escribe en el ordenador.
 En el caso de que se quieran utilizar comillas dobles se han de
 escribir dos caracteres ‘comilla simple’ seguidos, esto es,
 ‘‘comillas dobles’’.
 Tambi ́en se ha de tener cuidado con los guiones: se utiliza un ́unico
 gui ́on para la separaci ́on de s ́ılabas, mientras que se utilizan
 tres guiones seguidos para producir un gui ́on de los que se usan
como signo de puntuacion --- como en esta oración.
\end{document}